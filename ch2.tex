\chapter{Resource Theories}

\exercise{2.5}
Is $(\R, \leq, 1, *)$ a symmetric monoidal preorder?

\solution
It is not a monoidal preorder, as for example $-5\leq -1$ and $-2\leq -1$, but $-5*-2\nleq -1*-1$.

\exercise{2.8}
Check that if $(M, *,e)$ is a commutative monoid then $(\textbf{Disc}_M,=,*,e)$ is a symmetric monoidal preorder.

\solution
Unitality, associativity, and symmetry come for free from the definition of a commutative monoid, so we only need to check monotonicity.  Since $x\leq y\iff x=y$ this is easy to check, as $x_1\leq y_1$ and $x_2\leq y_2$ implies $x_1=y_1$ and $x_2=y_2$ which in turn implies that $x_1*x_2 = y_1 * y_2$.

\newpage

\exercise{2.20}
Formally prove that $t\leq v+w, w+u\leq x+z, v+x\leq y$ implies $t+u\leq y+z$.  Be explicit about where reflexivity and transitivity are used, and why symmetry need not be used.

\solution
In the below, R=reflexivity, M=monotonicity, A=associativity, T=transitivity.  Symmetry is not used, which you can tell from the diagram from the fact that no wires cross.
\begin{align}
	t\leq v+w, u\leq u &&\implies&& t+u\leq (v+w)+u = v+(w+u) &\quad& \textrm{[R, M, A, T]}\label{1} \\
	(\ref{1}),w+u\leq x+z &&\implies&& v+(w+u)\leq v+(x+z) &\quad& \textrm{[M]}\label{2} \\
	(\ref{1},\ref{2}) &&\implies&& t+u\leq v+(x+z) = (v+x)+z &\quad& \textrm{[T, A]}\label{3} \\
	v+x\leq y, z\leq z &&\implies&& (v+x)+z\leq y+z &\quad& \textrm{[M]}\label{4} \\
	(\ref{3},\ref{4}) &&\implies&& t+u\leq y+z &\quad& \textrm{[T]}\label{5}
\end{align}

\exercise{2.21}
Skipped.

\exercise{2.29}
Consider $(\B,\leq)$ with monoidal product $\lor$.  What's the monoidal unit?  Does it satisfy the rest of the conditions?

\solution
The monoidal unit should be $\texttt{false}$.  Discussed why in person (i.e. truth table)

\exercise{2.31}
Show that there is a monoidal structure on $(\N, \leq)$ where the monoidal product is standard $*$.  What should the monoidal unit be?

\solution
The monoidal unit should be 1.  We will show monotonicity as the other conditions are obvious.  If $x_1\leq y_1$ and $x_2\leq y_2$, then there are $a_1,a_2\in\N$ such that $y_1 = x_1+a_1$ and $y_2=x_2+a_2$.  Then
$$y_1 * y_2 = (x_1 + a_1) * (x_2 + a_2)= x_1*x_2+a_1*x_2 + a_2*x_1 + a_1*a_2\geq x_1*x_2.$$

\exercise{2.33}
Consider the divisibility order $(\N, |)$.  Does $0$ as monoidal unit and $+$ as monoidal product satisfy the conditions?

\solution
It does not, as for example $2|4$ and $1|1$, but $(2+1)\nmid(4+1)$, so monotonicity fails.

\exercise{2.34}
Consider the preorder $\textbf{NMY}$ with Hasse diagram $\texttt{no}\to\texttt{maybe}\to\texttt{yes}$, monoidal unit $\texttt{yes}$ and ``min'' as the monoidal product.  Define what ``min'' should be and check that the axioms hold.

\solution
\begin{tabular}{c|ccc} 
min & no & maybe & yes \\
\hline no & no & no & no \\
maybe & no & no & maybe \\
yes & no & maybe & yes
\end{tabular}

\exercise{2.35}
Let $S$ be a set and let $P(S)$ be its power set, with the subset relation as order.  Does $P(S)$ with unit $S$ and product given by set intersection satisfy the conditions of symmetric monoidal preorder?

\solution
The other conditions are easy to show, so we will show monotonicity only.  Let $A_1, A_2, B_1, B_2\subseteq S$ where $A_1\subseteq B_1$ and $A_2\subseteq B_2$.  Then $A_1\cap A_2 \subseteq A_1\subseteq B_1$ and $A_1\cap A_2\subseteq A_2\subseteq B_2$, which implies that $A_1\cap A_2\subseteq B_1\cap B_2$.

\exercise{2.36}
Let $\texttt{Prop}^\N$ denote the set of all mathematical statements one can make about a natural number, where we consider two statements to be the same if one is true if and only if the other is true.  Given $P,Q\in\texttt{Prop}^\N$, we say $P\leq Q$ if for all $n\in\N$, whenever $P(n)$ is true, so is $Q(n)$.  Define a monoidal unit and product on $\texttt{Prop}^\N$.

\solution
We define the monoidal unit to be the statement ``$n$ is a natural number'' (i.e. a statement that's always true) and the monoidal product to be logical AND.  Note that this SMP effectively reduces to the previous example, where we consider subsets $P = \{n\in\N\ | P(n)\}$.

\exercise{2.39}
Complete the proof of Proposition 2.38.

\solution
These conditions are inherited from the original SMP, so we have decided this problem is dumb.

\exercise{2.40}
What is $\textbf{Cost}^\textrm{op}$ as a preorder?  What is the monoidal unit and product?

\solution
$\textbf{Cost}^\textrm{op}$ is the same as $\textbf{Cost}$ but using $\leq$ rather than $\geq$.  From Proposition 2.38, we know that $\textbf{Cost}^\textrm{op}$ can use the same monoidal unit and product as the original, i.e. $0$ and $+$.

\exercise{2.43}
Check that the map $g:(\B, \leq, \true, \land)\to([0,\infty],\geq,0,+)$ with $g(\false) = \infty$ and $g(\true)=0$ is monoidal monotone.  Is $g$ strict?

\solution
Clearly $g(\false)\leq g(\true)$ as $\infty\geq 0$, so $g$ is monotone.  In addition $g(\true)=0$, so the function preserves identities exactly.  Finally note that $g(\false)+g(\false) = \infty + \infty = \infty = g(\false)=g(\false\land\false)$, the other products are trivial as we've shown the identity is preserved.  Hence $g$ is strict monoidal monotone.

\exercise{2.44}
Let $\textbf{Bool}$ and $\textbf{Cost}$ be as above, and consider $d,u:[0,\infty]\to\B$ as follows:
$$d(x):=\left\{\begin{array}{ll}\text { false } & \text { if } x>0 \\ \text { true } & \text { if } x=0\end{array} \quad\quad\quad u(x):=\left\{\begin{array}{ll}\text { false } & \text { if } x=\infty \\ \text { true } & \text { if } x<\infty\end{array}\right.\right.$$
Is $d$ monotonic, monoidal, and/or strict?  Is $u$?

\solution

\exercise{2.45}
\begin{enumerate}
	\item Is $(\N, \leq, 1, *)$ a monoidal preorder?
	\item If not, why not?  If so, does there exist a monoidal monotone $(\N,\leq, 0, +)\to (\N, \leq, 1,*)$?  If not, why not?
	\item Is $(\Z,\leq, *, 1)$ a monoidal preorder?
\end{enumerate}

\solution

