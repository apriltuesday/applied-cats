\chapter{Databases}

\exercise{3.3}
Discussed in person.

\exercise{3.9}
Check that $\Free(G)$ is a category for any graph $G$.

\solution
It is clear that concatenating any path with the trivial path yields the original, so unitality is satisfied.  Concatenation of paths is also associative (you can think of paths as lists of vertices/edges, which makes this clear).

\exercise{3.10}
Discussed in person

\exercise{3.12}
\begin{enumerate}
	\item What is the category $\textbf{1}$?
	\item What is the category $\textbf{0}$?
	\item What is the formula for the number of morphisms in $\textbf{n}$ for arbitrary $n\in\N$?
\end{enumerate}

\solution
\begin{enumerate}
	\item $\textbf{1}$ has a single object and only the identity morphism; it corresponds to the free category on the graph with one vertex.
	\item $\textbf{0}$ has no objects and no morphisms.
	\item The number of morphisms is $\sum_{i=1}^n i = \frac{n(n+1)}{2}$.  This is because there is one path of length $n-1$, two of length $n-2$, and so on until you get $n$ paths of length 1.
\end{enumerate}

\exercise{3.15}
In the loop graph, we identified paths with numbers $n\in\N$.  Given $m,n\in\N$, what number corresponds to the concatenation of their associated paths?

\solution
The concatenation corresponds to $m+n$.

\exercise{3.16}
\begin{enumerate}
	\item Write down the 10 paths in the free square category.
	\item Name two distinct parallel paths.
	\item Name two paths that are not parallel.
\end{enumerate}

\solution
\begin{enumerate}
	\item $\id_A, \id_B, \id_C, \id_D, f, g, h, i, f\fcmp h, g\fcmp i$
	\item $f\fcmp h$ and $g\fcmp i$
	\item $f$ and $g$
\end{enumerate}

\exercise{3.17}
See text.

\solution
This has the same morphisms as the commutative square, so $\{A, B, C, D, f ,g, h, i, f\fcmp h\}$.

\exercise{3.19}
See text.

\solution
$\{z, s, s\fcmp s, s\fcmp s\fcmp s\}$

\exercise{3.21}
See text.

\solution
$G_1$: $f=g$

$G_2$: $f=\id$

$G_3$: $f\fcmp h=g\fcmp i$

$G_4$: none

\exercise{3.22}
What is the preorder reflection of the category $\N$?

\solution
The preorder reflection of $\N$ is just the category with one object and its identity morphism.

\exercise{3.25}
Discussed in person.

\exercise{3.30}
\begin{enumerate}
	\item What is the inverse $f^{-1}:\underline{3}\to A$ of the function $f$ given in Example 3.29?
	\item How many distinct isomorphism are there $A\to\underline{3}$?
\end{enumerate}

\solution
\begin{enumerate}
	\item $f^{-1}(1) = b, f^{-1}(2)=a, f^{-1}(3)=c$
	\item This is just the number of permutations of a set of three elements, namely $3!=6$.
\end{enumerate}

\exercise{3.31}
Show that in any category $\mcC$, for any given object $c\in\mcC$, the identity $\id_c$ is an isomorphism.

\solution
By definition $\id_c\fcmp\id_c=\id_c$, so it's clearly an isomorphism.

\exercise{3.32}
A monoid in which every morphism is an isomorphism is a group.
\begin{enumerate}
	\item Is the monoid in Example 3.13 a group?
	\item Is the monoid in Example 3.18 a group?
\end{enumerate}

\solution
\begin{enumerate}
	\item $\N$ is not a group, as for example $s$ is not an isomorphism.  This is because $s$ composed with any element of the set of paths $\{z,s, s\fcmp s,\dots\}$ yields the set $\{s, s\fcmp s, s\fcmp s\fcmp s,\dots\}$, of which $z$ is clearly not a member.
	\item $\mcC$ is a group.  There are only two morphisms: the identity $z$ which is an isomorphism, and $s$ where $s\fcmp s=z$ making it an isomorphism as well.
\end{enumerate}

\exercise{3.33}
Let $G$ be a graph and $\Free(G)$ the corresponding free category.  Is it true that the only isomorphism in $\Free(G)$ are the identity morphisms?

\solution
This is true, since there are no equations, so for any $f:A\to B$ with an accompanying $g:B\to A$, we don't necessarily have $f\fcmp g=\id_A$ or $g\fcmp f=\id_B$.

\exercise{3.37}
Find all the functors from $\textbf{2}\to\textbf{3}$.

\solution
This is almost the number of functions that we found in Exercise 3.25 but without any that flip elements, which leaves 6.

\exercise{3.39}
Say where each of the 10 morphisms in $\mcF$ is is sent under the functor $F$ from Example 3.38.

\solution
$F(\id_{X'})=\id_{X}$ for any $X\in\{A,B,C,D\}$

$F(y')=y$ for any $y\in\{f,g,h,i\}$

$F(f'\fcmp h')=F(g'\fcmp i')=f\fcmp h$

\exercise{3.40}
See text.

\solution
One functor takes the arrow in $\mcC$ to the top arrow in $\mcD$, the other takes it to the bottom arrow.

\exercise{3.43}
Show that there is a category $\Cat$, where the objects are categories and morphisms are functors.

\solution
First we define $\id_\mcC:\mcC\to\mcC$ as follows: for any $c\in\Ob(\mcC)$, $\id_\mcC(c)=c$ and for any $f\in\mcC(c,d)$, $\id_\mcC(f)=f\in\mcC(\id_\mcC(c),\id_\mcC(d))=\mcC(c,d)$.  This is a functor, because $\id_\mcC(\id_c) = \id_c=\id_{\id_\mcC(c)}$, and for any $f\in\mcC(c_1,c_2)$ and $g\in\mcC(c_2,c_3)$. we have $\id_\mcC(f\fcmp g)=f\fcmp g = \id_\mcC(f)\fcmp\id_\mcC(g)$.

Next we define composition of functors $F:\mcC\to\mcD$ and $G:\mcD\to\mcE$ as follows: for any $c\in\Ob(\mcC)$, $(F\fcmp G)(c) = G(F(c))$ and for any $f\in\mcC(c,d)$, $(F\fcmp G)(f) = G(F(f))$.  Then $F\fcmp G:\mcC\to\mcE$ is a functor due to the functoriality of $F$ and $G$.  In particular, we have
$$(F\fcmp G)(\id_c)=G(F(\id_c))=G(\id_{F(c)})=\id_{G(F(c))}=\id_{(F\fcmp G)(c)},$$
as well as
$$(F\fcmp G)(f\fcmp g)=G(F(f\fcmp g)) = G(F(f)\fcmp F(g)) = G(F(f))\fcmp G(F(g)) = (F\fcmp G)(f)\fcmp (F\fcmp G)(g).$$

Note that this composition is associative simply because function composition is associative.  Unitality holds because $\id_\mcC$ is the identity function on objects and morphisms, so $(\id_\mcC\fcmp F)(c)=F(c)$ and $(\id_\mcC\fcmp F)(f)=F(f)$ and similarly for $F\fcmp\id_\mcD$.  Therefore $\Cat$ is indeed a valid category.







